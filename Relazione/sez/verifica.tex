\newpage
\section{Validazione}
 

\subsection{Scelte progetturali}

\subsection{Strumenti automatici}
Per garantire che il sito sia correttamente visualizzato e che rimanga accessibile nel maggior numero di browser possibili si è verificata la validità di tutte le pagine tramite l'utilizzo di molteplici validator e sono stati effettuati dei test di visualizzazione subrowser meno recenti
\subsubsection{W3C Validator plugin for Visual Studio Code}

\subsubsection{Markup Validation Service w3.org}
Tutte le pagine dell'utente generico sono state validate attraverso questo validatore. 
\subsubsection{W3C CSS Validator w3.org}
fogli di stile 
\subsubsection{Vamolà}
Questo strumento è stato utilizzato per la validazione secondo le linee Guida: WCAG 2.0 (Level AAA)
\subsubsection{TotalValidator}
Tutte le pagine sono state validate utilizzando Total Validator, 
\subsubsection{WAVE Accessibility Extension}
 WAVE, estensione per Firefox.\\

\subsubsection{ColorSafe}
Per la generazione di una palette di colori accessibile alla maggior parte dei disturbi visivi è stato scelto di utilizzare questo sito (\url(http://colorsafe.co/)) che, a partire da un colore di background, rende disponibili alcuni colori secondo gli standard WCAG 2.0 Level AAA.
\subsection{Test effettuati}
Il sito è stato testato con successo su Chrome ecc.. 
\\Inoltre sono stati effettuati dei test anche per la versione mobile, in particolare su:
\begin{itemize}
	\item Lista modelli telefoni sui quali si è provato il sito;
\end{itemize}
