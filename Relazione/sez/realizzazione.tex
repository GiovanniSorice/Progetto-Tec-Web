\newpage
\section{Realizzazione}
Per la realizzazione del sito si è fatto riferimento ad alcune linee guida che ci hanno permesso di impostare il lavoro fin da subito, permettendo una scrittura agevole e relativamente semplice del codice.

\subsection{Responsive Design}
TV Hunter è stato sviluppato con un approccio \textbf{mobile first} responsive.\\
Sono presenti tre breakpoint che permettono una visualizzazione di qualità delle pagine da tutti i tipi di dispositivi comunemente usati per la navigazione.\\
I breakpoint (indicati con l'intera \textit{media query}) sono i seguenti:
\begin{itemize}
	\item \textbf{@media screen and (min-width: 751px) and (max-width: 1024px)} delimita la modalità tablet. Adatta a schemi medio-grandi di tablet o a schermi grandi di smartphone.
	\item \textbf{@media screen and (min-width: 1025px)} delimita l'inizio della modalità desktop. 
\end{itemize}
Lo stile per i dispositivi mobili con larghezza massima di 750px è specificato all'esterno di media query, questà è una caratteristica del codice mobile first.\\
È presente una ulteriore media query per la modalità di stampa.
\subsubsection{Layout di stampa}
Per permettere all'utente di stampare pagine del sito è stato definito un layout di stampa che elimina dalla pagina elementi non utili alla stampa.\\
Ad esempio non sono visualizzati l'header, la barra di navigazione superiore ed inferiore. Gli sfondi scuri sono sostituiti dal colore bianco e i testi sono convertiti in colore nero. Le miniature delle serie TV non sono visualizzate nelle pagine di \textit{esplora} e \textit{preferiti} per evitare il consumo eccessivo di colore in fase di stampa.

\subsection{Linguaggi e strumenti}

\subsubsection{XHTML 1.0 Strict}
Html 5, versione Html 

\subsubsection{PHP}
L'utilizzo di PHP è stato esteso soprattutto nella parte dinamica delle pagine, ed è stato in buona parte preferito rispetto a JavaScript per il fatto che esso risiede nel server e non richiede prerequisiti da parte del client che poi vuole fruire la pagina.

\subsubsection{JavaScript}
Utilizzato 

\subsubsection{Struttura}
Tutta la parte strutturale è stata mantenuta all'interno dei files html oppure php dentro la cartella HTML. La parte presentazionale invece è stata posta all'interno della cartella CSS, le immagini all'interno della cartella images, mentre tutti gli altri file di utilità sono stati posti in cartelle più specifiche. La struttura scelta dal gruppo è quella gerarchica, in quanto tutte le informazioni sono disponibili a partire da una barra di navigazione sempre visibile.

\subsection{Presentazione}

\subsection{Comportamento}

Per quanto riguarda la parte dell'utenza generica si è deciso di utilizzare quanto meno possibile elementi JavaScript, prediligendo piuttosto la parte php. Questo ha permesso un comportamento quasi completamente predicibile - in quanto sottoposto a meno variabili di contorno dettate dal client - ed è stato sfruttato principalmente per mostrare a schermo le informazioni contenute nel database.
\\Per quanto riguarda invece la parte dell'amministrazione entrambi i linguaggi sono stati sfruttati quando necessario per poter generare una fruizione semplificata del sito da parte di un amministratore, evitando i frequenti caricamenti di pagina inevitabilmente generati da una gestione php-side. È stata inoltre predisposta la sanitizzazione e la validazione degli input da parte dell'utente, con una discreta gestione degli errori. Da notare anche l'adozione di un sistema anti-SQL Injection per la compilazione dei form in MySQL.
